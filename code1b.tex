To analyze the right tail of the dataset the following code hase been used. Libraries \textbf{ismev} and \textbf{evir} have been used.
\begin{lstlisting}
#going back to non-rescaled values
vals = 35*vals

par(mar=c(2,2,2,2))
#plotting the confidence intervals for paratemters of fittend 
#GPD changing the threshold
gpd.fitrange(vals, umin=300, umax=550)
title("Confidence interval for shape and scale varying threshold")
#considering this plot and the size our sample we decide to cut in 380
my_threshold = 380

ecdf(vals)(my_threshold)
#the percentage of dataset above the threshold is the 20%
my_gpd<-gpd(vals,my_threshold)
my_gpd$par.ests

#Save the parameters obtained with maximum likelihood
gamma = my_gpd$par.ests[1]
beta  = my_gpd$par.ests[2]

#let us evaluate the goodness of fit with a q-q plot
qplot(vals, threshold=my_threshold,gamma)

#plot the distribution over the histogram of the tail
x = seq(0,max(vals)-my_threshold, length=100)
hist(vals[vals>my_threshold], freq=FALSE,+
    breaks=seq(my_threshold,780,30), main="Histogram of values over threshold", legend="GPD approximation")
lines(x+my_threshold,dGPD(x, gamma, beta), col='green')
\end{lstlisting}

Using the generalized Pareto distribution fitted with the script above, the quantile of $1-1/2n$, where n is the number of observations in the dataset, has been obtained with the following implementation
\begin{lstlisting}
#evaluation of quantile
n = length(vals)
p = 1-1/(2*n)
my_p = -((1-p)*length(vals)/length(vals[vals>my_threshold])-1);
quantile_p = my_threshold+qGPD(my_p,gamma,beta)
#the empirical closest quantile we have is:
quantile(vals, p)
\end{lstlisting}

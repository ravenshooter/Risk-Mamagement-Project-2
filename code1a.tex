\begin{lstlisting}

library(stats4)
library(MASS)
library(gPdtest)
library(ggplot2)

data = read.csv('winter-negative-temperature-sum-.csv')
vals = as.matrix(data[,2])

n = length(vals)

# Plot the time series
data_overview = data.frame(year=data[,1], temp=data[,2])

# Plot histogram, mean and median
ggplot(data_overview, aes(x=year, y=temp)) + geom_line()
(ggplot(data_overview, aes(x=temp)) 
  + geom_histogram()
  + geom_vline(xintercept = mean(vals), col="Red")
  + geom_vline(xintercept = median(vals), col="Blue"))


################# FIT DIFFERENT DISTRIBUTIONS #######################

# Scale the values for fitting the Chi Square,
#doesn't impact the other distributions
vals = vals/35

# Fit a normal around the data
fittednorm = fitdistr(vals, "normal")$estimate
qqplot(qnorm((n-seq(1,n)+1)/(n+1))*fittednorm[2]+fittednorm[1], vals,
ylab="Empirical Quantiles", xlab="Normal Quantiles")
abline(0,1)

# Fit an exponential around the data
fittedexp = fitdistr(vals, "exponential")$estimate
qqplot(qexp((n-seq(1,n)+1)/(n+1), rate=fittedexp[1]), vals,
ylab="Empirical Quantiles", xlab="Exponential Quantiles")
qqline(vals, distribution = function(q){qexp(q,rate=fittedexp[1])})

# Fit a weibull around the data
fittedw<-fitdistr(vals, "weibull")$estimate
qqplot(qweibull((n-seq(1,n)+1)/(n+1),shape=fittedw[1], scale=fittedw[2]), vals, ylab="Empirical Quantiles", xlab="Weibull Quantiles")
abline(0,1)

# Fit a chi square around the data
fittedchi<-fitdistr(vals, "chi-squared", start=list(df=2))$estimate
qqplot(qchisq((n-seq(1,n)+1)/(n+1), df=fittedchi[1]), vals, ylab="Empirical Quantiles", xlab="Chi-Square quantiles")
abline(0,1)


# Plot a histogram and the different theoretical distributions
x = seq(0,30, length=100)
hist(vals, freq=FALSE, breaks=20)
lines(x,dchisq(x, df=fittedchi[1]), col='green')
lines(x,dweibull(x, shape=fittedw[1], scale=fittedw[2]), col='blue')
lines(x,dnorm(x, mean=fittednorm[1], sd=fittednorm[2]), col='red')
lines(x,dexp(x, rate=fittedexp[1]), col='yellow')

\end{lstlisting}
